\documentclass{article}
\usepackage[UTF8]{ctex}
\usepackage{graphicx}
\usepackage[colorlinks, linkcolor=blue]{hyperref}
\begin{document}
\begin{center}
    \includegraphics[width=0.23\textwidth]{logo.png} \\
    项目名称:\underline{提取debian软件包的控制信息生成SPEC文件} \\
    导师:\underline{王铮}  \\
    申请人:\underline{杨凌云} \\
    邮箱:\underline{tinaaaaa@sjtu.edu.cn} \\
    日期:\underline{\today}
\end{center}
\tableofcontents
\newpage

\section{项目背景}
\subsection{项目简介}
为 openEuler 引入debian的软件包,需要把deb形式的软件包,转成rpm形式的软件包。而生成rpm软件包, spec文件是必须项。

\subsection{项目产出要求}
\begin{itemize}
    \item 生成SPEC文件需要符合规范,重点关注多子包的情况
    \item 软件通过SPEC文件能够在openEuler完成构建
\end{itemize}

\subsection{项目链接}
\subsubsection{参考资料}
\begin{itemize}
    \item \href{https://www.debian.org/doc/debian-policy/ch-controlfields.html}
    {https://www.debian.org/doc/debian-policy/ch-controlfields.html}
    \item \href{https://www.debian.org/doc/debian-policy/ch-source.html\#main-building-script-debian-rules}
    {https://www.debian.org/doc/debian-policy/ch-source.html\#main-building-script-debian-rules}
    \item \href{https://rpm-packaging-guide.github.io/\#what-is-a-spec-file}
    {https://rpm-packaging-guide.github.io/\#what-is-a-spec-file}
\end{itemize}

\subsubsection{成果仓库}
\href{https://gitee.com/openeuler/oec-application}
{https://gitee.com/openeuler/oec-application}

\section{技术可行性分析}
\subsection{Linux相关}
我从半年前配置CSAPP环境时开始使用Ubuntu系统,目前使用的主力系统是Arch Linux,因此我对Linux系统以及各大发行版
有较多了解,对Linux的命令和脚本较为熟悉。

\subsection{软件包相关}
使用Ubuntu系统时,我对apt命令下载软件包很有兴趣,了解过软件包的结构,知道软件包的安装构建过程。

\subsection{python相关}
我在大一开始时自学了python,并在自学国外课程时通过python写爬虫爬取资料 (比如MIT微积分网站上的数百个pdf文件)。
其中使用到的正则表达式等知识同样可以运用到本项目中。

\section{项目基本知识}
\label{sec:basic_knowledge}
这部分内容我在阅读了参考资料后自己进行了总结,对应地整理出了三份文档存放在我的
\href{https://github.com/tinaaaaa42/deb2rpm}{项目仓库}中。

下面我进一步地介绍一下这三份文档的内容。

\subsection{deb包的control文件}
\label{sec:deb_control}
deb包的control文件是一个文本文件,包含了软件包的基本信息,比如软件包的名称,版本,作者,依赖等等。
它既会在源包中出现,也会在二进制包中出现。其基本格式为若干个键值对。
其中有如下重要的字段:

\begin{itemize}
    \item Package  软件包的名称
    \item Source   源包的名称
    \item Version  软件包的版本
    \item Architecture  软件包的架构
    \item Depends  软件包的依赖
    \item Description  软件包的描述
    \item Maintainer  软件包的维护者
    \item Homepage  软件包的主页
    \item Section  软件包的分类
    \item Priority  软件包的优先级
\end{itemize}

\subsection{deb包的rule文件}
\label{sec:deb_rule}
deb包的rule文件是一个脚本文件,包含了软件包的构建过程,比如软件包的编译,安装,打包等等。
其基本格式为若干个规则,每个规则包含了若干个目标和依赖。
其中典型的有:
\begin{itemize}
    \item build 构建软件包
    \item build-arch 构建软件包(特定架构)
    \item build-indep 构建软件包(架构无关)
    \item clean 清理软件包
    \item install 安装软件包
    \item binary 构建二进制包
    \item binary-arch 构建二进制包(特定架构)
    \item binary-indep 构建二进制包(架构无关)
    
\end{itemize}

\subsection{rpm包的spec文件}
rpm包的spec文件是一个文本文件,包含了rpm软件包的基本信息,比如软件包的名称,版本,作者,依赖等等。
其基本格式为\emph{前言段}和\emph{主体段},前者包含包的元数据,类似\ref{sec:deb_control}中的control文件;
后者包含构建等过程的脚本,类似\ref{sec:deb_rule}中的rule文件。

\emph{前言段}中有如下重要的段落:
\begin{itemize}
    \item Name  软件包的名称
    \item Version  软件包的版本
    \item Release  软件包的发布号
    \item Summary  软件包的简介
    \item License  软件包的许可证
    \item URL  软件包的主页
    \item Source0  软件包的源码
    \item Patch0  软件包的补丁
    \item BuildArch  软件包的架构
    \item BuildRequires  软件包的构建依赖
    \item Requires  软件包的运行依赖
\end{itemize}

\emph{主体段}中有如下重要的段落:
\begin{itemize}
    \item \%description  软件包的描述
    \item \%prep  构建前准备
    \item \%build  构建阶段
    \item \%install  安装阶段
    \item \%check  测试阶段
    \item \%files  文件列表
    \item \%changelog  修改日志
\end{itemize}

\section{项目实现}
根据\ref{sec:basic_knowledge}中的知识,可以发现deb包和rpm包的构建过程有很多相似之处。
比如rpm包的spec文件中的\emph{前言段}主要包含元数据,可以与deb包中的control文件对应;
spec文件中的\emph{主体段}主要包含构建脚本等内容,可以与deb包中的rule文件对应。
更具体的,我们可以在将spec前言段和control文件在字段层面进行对应,如下表所示:
\begin{center}
    \begin{tabular}{|c|c|}
    \hline
    rpm spec前言段 & deb control\\
    \hline
    Name & Package \\
    \hline
    Version & Version或Standards Version \\
    \hline
    Release & Version \\
    \hline
    Summary & Description \\
    \hline
    License & License \\
    \hline
    URL & Homepage \\
    \hline
    Source0 & Source \\
    \hline
    BuildArch & Architecture \\
    \hline
    BuildRequires & Build-Depends \\
    \hline
    Requires & Depends \\
    \hline
    \end{tabular}
\end{center}

同样的,我们可以在将spec主体段和rule文件在规则层面进行对应,如下表所示:
\begin{center}
    \begin{tabular}{|c|c|}
        \hline
        rpm spec主体段 & deb rule \\
        \hline
        \%description & Description \\
        \hline
        \%prep & Source, Build \\
        \hline
        \%build & build \\
        \hline
        \%install & install \\
        \hline
        \%check & test \\
        \hline
        \%files & source \\
        \hline
    \end{tabular}
\end{center}
另外, rpm包的spec文件中的\emph{\%changelog}可以通过deb包中的\emph{.change文件}来提取。

\section{时间安排}
\subsection{研发阶段(7.1-8.15)}
\begin{itemize}
    \item 完善deb包和rpm包的对应关系
    \item 完成deb包和rpm spec文件的转换工具
    \item 在openEuler上对一些基础包进行测试
\end{itemize}

\subsection{测试阶段(8.16-9.30)}
\begin{itemize}
    \item 在openEuler上对更多包进行测试,并完善细节处理
    \item 完成对包转换工具的测试,并对其进行优化
    \item 完成文档的书写
\end{itemize}

\end{document}